\documentclass[12pt]{article}
\usepackage[margin=2cm,bottom=2cm]{geometry}
\usepackage[utf8]{inputenc}
\usepackage{hyperref}

\begin{document}
 \title{THE DINING PHILOSOPHERS PROBLEM}
 \author{Furkan Karakoyunlu, 112200036}
 \maketitle
 
 \section*{About the Problem}
 \begin{flushleft}
  The \textbf{dining philosophers} problem is invented by the Dijkstra. In this problem we have five philosophers who can do only
  thinking and eating. In the room there is a circular table with five chairs. The table has five plates for the philosophers. 
  However there are only five forks available. Philosophers can eat only with two forks. Each philosopher thinks and when s/he 
  gets hungry, eats the food with two forks which are closest to the philosopher. If a philosopher tooks the forks and eats 
  for a while s/he puts down the forks and starts thinking.\\
  \bigskip
  Philosophers have a thinking process for a while and they can not talk to each other.
 \end{flushleft}
 
 \section*{Main Problems}
 \begin{flushleft}
  If all the philosophers take the right or left forks, they all have one fork and can not eat. If they try to pick the nearest 
  two fork, which one takes the two fork first can eat the food. In this situation there is no guarantee.\\
  \bigskip
  If all philosophers take one fork, they can not eat or they all put down the forks in order to other philosopher can eat, 
  again they can not eat. These problems called \emph{starvation}.\\
  \bigskip
  According to these situations every philosopher has a possibility to eat but it can not be guaranteed. For instance, one 
  of the philosopher may remain hungry in all cases.\\
  \bigskip
  Another big problem is deadlocks. If the each philosopher performs$^1$;
  \begin{itemize}
   \item think until the left fork is available; when it is, pick it up;
   \item think until the right fork is available; when it is, pick it up;
   \item when both forks are held, eat for a fixed amount of time;
   \item then, put the right fork down;
   \item then, put the left fork down;
   \item repeat from the beginning.
  \end{itemize}
  these actions causes \emph{deadlock}. For instance, a philosopher who took a fork, waits for another philosopher to put down the fork
  to put down the forks. (It seems complicated but it is not.) It locks the system.
 \end{flushleft}
 \pagebreak
 \section*{Solutions}
 \subsection*{Arbitrator Solution}
  \begin{flushleft}
   This solution guarantee that a philosopher can pick up two forks or none by asking permission from arbitrator. We can 
   think arbitrator as a waiter. In order to pick up the forks a philosopher must ask permission. If the waiter gives 
   the permission, philosopher can eat. The waiter gives permission to only one philosopher at a time. If a philosopher 
   is busy with eating and one of the other philosopher is requesting to pick up forks s/he must wait until other philosophers 
   request is done. We can implement the waiter as a \underline{mutex}. 
  \end{flushleft}
 
 \subsection*{Resource Hierarchy Solution}
 \begin{flushleft}
  This solution is the originally proposed by Dijkstra. It assigns numbers to forks 1 to 5. Philosophers can pick up the 
  lower numbered forks first. And the order in which each philosopher puts down the forks does not matter. In this situation 
  if four philosopher pick up their lower numbered fork simultaneously, only the fork which is number 5 will remain on the 
  table. Thus the fifth philosopher will not be able to pick up any fork. It can be listed as$^2$:
  \begin{itemize}
   \item philosopher\#1 picks up fork\#1
   \item philosopher\#2 picks up fork\#2
   \item philosopher\#3 picks up fork\#3
   \item philosopher\#4 picks up fork\#4
   \item philosopher\#5 can not pick up fork\#5! Because s/he will need two forks and needs to pick up the lower numbered 
   fork first.
  \end{itemize}
  So the fork\#5 goes to philosopher\#4 - \emph{no deadlocks!}\\
  If we have fork\#3 and \#5. We decide we need fork\#2. Forks\#3 and \#5 are higher numbers. So we will have to do:
  \begin{itemize}
   \item put down fork\#5
   \item put down fork\#3 (the order you put these down in does not matter)
   \item pick up fork\#2
   \item pick up fork\#3
   \item pick up fork\#5
  \end{itemize}
  It's slow!
 \end{flushleft}
 \pagebreak
 \subsection*{Chandy/Misra Solution}
 \begin{flushleft}
  This solution removes the central decision mechanism. However it violates the ``philosophers do not speak with each other" 
  rule. The solution consists of these steps:
  \begin{itemize}
   \item[1] Produce a fork for a pair of philosophers. Give this fork to the lower numbered philosopher. Every fork can 
   be clean or dirty. Initially every fork is dirty.
   \item[2] If a philosopher wants to eat s/he must use neighbor forks and sends a request for forks that s/he does not have.
   \item[3] If a philosopher gets a request, s/he checks the status of the forks and if they are clean s/he continues 
   to use them but if they are dirty, s/he puts down the forks.
   \item[4] Philosopher marks the fork as a dirty after eating. However, if another philosopher requested the forks, s/he 
   cleans the forks and then puts them down.
  \end{itemize}
  We can explain this algorithm in this way, firstly we add a status to forks. According to this a fork can be clean or dirty. And it is 
  decided with demands. If there is no request to a fork, it stays clean but if there is a request on a fork it's status 
  changed to dirty. In this case we prevent another philosopher's request. Every fork can be used by philosopher who requested 
  it, so there is an assignment to philosophers who will use the forks. Beside, in order to complete fork request fork assignments 
  must be done. 
 \end{flushleft}
 \pagebreak
 \begin{thebibliography}{9}
  \bibitem{wikipedia}
    \url{https://en.wikipedia.org/wiki/Dining_philosophers_problem}
  \bibitem{adit.io}
    \url{http://adit.io/posts/2013-05-11-The-Dining-Philosophers-Problem-With-Ron-Swanson.html}
  \bibitem{bilgisayarkavramlari}
    \url{http://bilgisayarkavramlari.sadievrenseker.com/2012/01/22/filozoflarin-aksam-yemegi-dining-philosophers/}
 \end{thebibliography}

\end{document}
